\section{Resultados}

Nesta seção, serão exibidos os resultados referentes aos operadores em cada tecnologia aplicada.

\begin{figure}[htbp]
	\centering
	\includegraphics[width=1\linewidth]{images/memory-operations.png}
	\caption{Comparação do tempo médio de execução dos processos de transferência de dados.}
	\label{fig:memory-operations}
\end{figure}

\begin{figure}[htbp]
	\centering
	\includegraphics[width=1\linewidth]{images/point-operations.png}
	\caption{Comparação do tempo médio de execução dos operadores pontuais.}
	\label{fig:point-operations}
\end{figure}

\begin{figure}[htbp]
	\centering
	\includegraphics[width=1\linewidth]{images/erosion-operations.png}
	\caption{Comparação do tempo médio de execução do operador de erosão com diferentes \textit{kernels} 3x3.}
	\label{fig:erosion-operations}
\end{figure}

\begin{figure}[htbp]
	\centering
	\includegraphics[width=1\linewidth]{images/convolution-3x3-operations.png}
	\caption{Comparação do tempo médio de execução do operador de convolução com diferentes \textit{kernels} 3x3.}
	\label{fig:convolution-3-operations}
\end{figure}

 \begin{figure}[htbp]
	\centering
	\includegraphics[width=1\linewidth]{images/convolution-5x5-operations.png}
	\caption{Comparação do tempo médio de execução do operador de convolução com diferentes \textit{kernels} 5x5.}
	\label{fig:convolution-5-operations}
\end{figure}

\subsection{Observações}

\begin{list}
    \item Na \Cref{fig:memory-operations} é possível identificar uma disparidade no processo de cópia
\end{list}
