\section{Resultados}

Nesta seção, serão exibidos os resultados referentes aos operadores em cada tecnologia aplicada.

\begin{figure}[!htb]
	\centering
	\includegraphics[width=1\linewidth]{images/memory-operations.png}
	\caption{Comparação do tempo médio de execução dos processos de transferência de dados.}
	\label{fig:memory-operations}
\end{figure}

\begin{figure}[!htb]
	\centering
	\includegraphics[width=1\linewidth]{images/point-operations.png}
	\caption{Comparação do tempo médio de execução dos operadores ponto-a-ponto.}
	\label{fig:point-operations}
\end{figure}

\begin{figure}[!htb]
	\centering
	\includegraphics[width=1\linewidth]{images/erosion-operations.png}
	\caption{Comparação do tempo médio de execução do operador de erosão com diferentes \textit{kernels} 3 $\times$ 3.}
	\label{fig:erosion-operations}
\end{figure}

\begin{figure}[!htb]
	\centering
	\includegraphics[width=1\linewidth]{images/convolution-3x3-operations.png}
	\caption{Comparação do tempo médio de execução do operador de convolução com diferentes \textit{kernels} 3 $\times$ 3.}
	\label{fig:convolution-3-operations}
\end{figure}

 \begin{figure}[!htb]
	\centering
	\includegraphics[width=1\linewidth]{images/convolution-5x5-operations.png}
	\caption{Comparação do tempo médio de execução do operador de convolução com diferentes \textit{kernels} 5 $\times$ 5.}
	\label{fig:convolution-5-operations}
\end{figure}

\subsection{Observações}

A seguir, serão exibidas observações referentes aos resultados obtidos.

Na \Cref{fig:memory-operations} é possível identificar uma disparidade no processo de cópia causado, provavelmente, por compartilhamento implícito de memória ou \textit{copy-on-write} (COW).

Na \Cref{fig:convolution-3-operations}, o operador de desfoque gaussiano da biblioteca OpenCV obtém um resultado inferior ao operador de convolução com \textit{kernel}.

Apenas a implementação em SYCL feita nesta pesquisa foi a única real beneficiada ao utilizar o \textit{kernel} e elemento estruturante separadamente, diferentemente dos resultados apontados em \cite{Dantas2016opencl}.