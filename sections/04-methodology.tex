\section{Metodologia}

Nesta seção, será apresentada a metodologia adotada para o desenvolvimento dos experimentos de benchmark comparativo entre diferentes tecnologias de computação paralela aplicadas ao processamento de imagens.

\subsection{Estudo e Experimentos}

A metodologia adotada será baseada nos experimentos de \cite{Dantas2015opencl} que utiliza a biblioteca VisionGL com operadores de processamento de imagens implementados em OpenCL como base para análise de desempenho de diferentes tecnologias de computação paralela. Repetiremos o experimento com estas tecnologias:
\begin{itemize}
	\item SYCL através do compilador AdaptiveCPP.
	\item CUDA com Julia através da biblioteca CUDA.jl.
	\item OpenCV com Python através da biblioteca opencv-python.
	\item OpenCL através da biblioteca VisionGL.
	\item MATLAB Parallel Computing Toolbox.
\end{itemize}

Desenvolveremos os operadores de processamento de imagens conforme descritos nas \Crefrange{eq:grayscale}{eq:dilation} nas tecnologias que buscamos analisar o desempenho.

A foi feito um \textit{fork} da biblioteca VisionGL\footnote{\url{https://github.com/ddantas/visiongl}} para estudo da estrutura do projeto e desenvolvimento dos operadores lógicos em outras tecnologias. O \textit{fork} este presente no repositório do GitHub\footnote{\url{https://github.com/jusqua/visiongl}}.

\subsection{Recursos Utilizados}

As especificações do ambiente de trabalho utilizado no momento da realização dos experimentos:
\begin{itemize}
	\item \textbf{Hardware:}
	      \begin{itemize}
		      \item Processador: Intel Core i9-14900K
		      \item Memória RAM: 32GB DDR5
		      \item Placa de Vídeo: NVIDIA GeForce RTX 5060 Ti 16GB
	      \end{itemize}
	\item \textbf{Software:}
	      \begin{itemize}
		      \item Sistema Operacional: Fedora Linux 42
		      \item Kernel: 6.16.7-200
		      \item GPU Driver: 580.82.07
		      \item CUDA Driver: 13.0.84
		      \item OpenCL: 3.0
		      \item Compilador C++: Clang 20.1.8-4
		      \item Compilador SYCL: AdaptiveCPP (commit fe13c2d1f1244d9a2faffeee0bb34e6bfec18aba)
		      \item Runtime Julia: 1.11.7
		      \item Runtime Python: 3.13.7
		      \item Runtime MATLAB: 25.1.0.2943329 (R2025a)
	      \end{itemize}
	\item \textbf{Bibliotecas:}
	      \begin{itemize}
		      \item VisionGL: (commit 45078178be564f59c601fca295afca9897b90dba)
		      \item OpenCV: 4.10.0
		      \item CUDA.jl: 5.8.3
	      \end{itemize}
\end{itemize}

\subsection{Estrutura do Benchmark}

Para garantir comparações justas entre as tecnologias, foi desenvolvida uma estrutura de benchmark que:
\begin{enumerate}
	\item Carrega a mesma imagem de entrada para todas as tecnologias.
	\item Realiza o aquecimento da GPU executando cada operador uma vez antes das medições.
	\item Executa cada operador múltiplas vezes (10000 iterações) para obter medições estatisticamente significativas.
	\item Mede o tempo de execução incluindo transferência de dados entre CPU e GPU.
	\item Mede o tempo de execução apenas do processamento no dispositivo.
	\item Registra uso de memória e outras métricas relevantes.
\end{enumerate}

A imagem de teste utilizada nos experimentos é a mostrada na \Cref{fig:fundus} com dimensões de $3504x2336$ píxeis e 3 canais de cor. A imagem foi escolhida por conter por ser grande o suficiente para ter um atraso significativo no processamento.

\begin{figure}[htbp]
	\centering
	\includegraphics[width=0.8\linewidth]{images/fundus.jpg}
	\caption{Imagem de olho com retinopatia diabética. Obtida em: \url{https://www5.cs.fau.de/research/data/fundus-images/}}
	\label{fig:fundus}
\end{figure}

Todos os experimentos foram executados com a mesma imagem de teste para avaliar o comportamento de escala de cada tecnologia. Os resultados foram salvos em formato estruturado para análise posterior.

Todos os experimentos e simulações deste trabalho que foram realizados estão presentes no repositório do GitHub\footnote{\url{https://github.com/jusqua/dip-benchmark}}.
