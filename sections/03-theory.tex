\section{Fundamentação Teórica}

Nesta seção, serão aprensentados conceitos e ferramentas que foram utilizadas no desenvolvimento da pesquisa. Serão abordados conceitos de paralelismo, arquiteturas heterogênias, operadores de processamento de imagens, modelos de programação paralela, em específico CUDA e SYCL, e brevemente sobre frameworks e bibliotecas relevantes para o tema. Esses conceitos mostram o estado da arte na área de computação paralela e processamento de imagens, fornecendo a base teórica necessária para compreender o desenvolvimento do trabalho.

\subsection{Programação paralela}

De forma simples, o principal objetivo da programação paralela é computar algo mais rapidamente, um processo é dado paralelo quando uma tarefa é dividida em pequenas partes que são executadas simultaneamente com o intuito de acelerar a execução desta tarefa~\cite{ArpaciDusseau23}, o que aproveita melhor os recursos disponíveis do dispositivo. A programação paralela é especialmente útil para tarefas que envolvem grandes volumes de dados ou cálculos complexos, onde a divisão do trabalho pode levar a melhorias significativas no desempenho.
