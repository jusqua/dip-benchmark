% --------------------IMPORTANTE ------------------------
% Usar compiliador pdfLatex no menu de configurações

\documentclass[12pt, a4paper]{article}
\usepackage[utf8]{inputenc}
\usepackage[brazilian]{babel}
\usepackage{amsmath}
\usepackage{amsfonts}
\usepackage{amssymb}
\usepackage{verbatim}
\usepackage[T1]{fontenc}
\usepackage{graphicx}
\usepackage{physics}
\usepackage{siunitx}
\usepackage{cite}
\usepackage[top=2.5cm, bottom=2.5cm, left=3cm, right=3cm]{geometry}
\usepackage{makeidx}
\usepackage{helvet}
\usepackage{graphicx}
\usepackage{sectsty}
\usepackage{tocbibind}% ddantas 24/3/2021
\usepackage{makecell}
%\usepackage{multirow}
\usepackage{adjustbox}
% to use \toprule, \midrule
\usepackage{booktabs}
\usepackage{indentfirst}
%\renewcommand{\lstlistingname}{Algorithm}% Listing -> Algorithm
%\usepackage[brazilian,hyperpageref]{backref}	 % Paginas com as citações na bibl
\usepackage{hyperref}
\usepackage[alf]{abntex2cite} % Citações padrão ABNT
\usepackage[utf8]{inputenc}
% Para colorir o texto
% ddantas 06/08/2021
\usepackage[dvipsnames]{xcolor}
\usepackage{verbatim}

\usepackage{amsmath}

% Pacote para a definição de novas cores
\usepackage{xcolor}
% Definindo novas cor
\definecolor{verde}{rgb}{0.25,0.5,0.35}
\definecolor{jpurple}{rgb}{0.5,0,0.35}
\definecolor{darkgreen}{rgb}{0.0, 0.2, 0.13}
%\definecolor{oldmauve}{rgb}{0.4, 0.19, 0.28}
% Configurando layout para mostrar codigos Java
%\documentclass{article}
\usepackage{listings}
\usepackage{xcolor}

\lstset{ frame=TB, numbers=left, stepnumber=1, basicstyle=\ttfamily\scriptsize,
captionpos=b, xleftmargin=0.7cm, xrightmargin=0.4cm, escapechar=@, breaklines=true,
%postbreak=\mbox{\textcolor{red}{$\hookrightarrow$}\space}, retirada do espaço em branco
}

%LSTLISTING USADO ANTERIOMENTE
\begin{comment}
\lstdefinestyle{mystyle}{
    backgroundcolor=\color{backcolour},
    commentstyle=\color{codegreen},
    keywordstyle=\color{magenta},
    numberstyle=\tiny\color{codegray},
    stringstyle=\color{codepurple},
    basicstyle=\ttfamily\footnotesize,
    breakatwhitespace=false,
    breaklines=true,
    captionpos=b,
    keepspaces=true,
    numbers=left,
    numbersep=5pt,
    showspaces=false,
    showstringspaces=false,
    showtabs=false,
    tabsize=2
}
\lstset{style=mystyle}
\end{comment}

% Para definir a fonte da seção
\sectionfont{\fontsize{12}{12}\selectfont}
\subsectionfont{\fontsize{11}{11}\selectfont}

\input{commands}
\renewcommand{\familydefault}{\sfdefault}
\usepackage{tocbasic}

\makeatletter
\renewcommand{\tableofcontents}{ \null\hfill\textbf{\Large\contentsname}\hfill\null\par
\@mkboth{\MakeUppercase\contentsname}{\MakeUppercase\contentsname} \@starttoc{toc}
}
\makeatother

\author{COPES - UFS}
\title{Modelo_Relatorio_Final}
\date{}

%%%%%%%%%%%%%%%%%%%%%%%%%%%%%%%%%%%%%%%%%%%%%%%%%%%%%%%%%%%%
%%%%%%%%%%%%%%%%%%%%%%%%%%%%%%%%%%%%%%%%%%%%%%%%%%%%%%%%%%%%
%% main text
%%%%%%%%%%%%%%%%%%%%%%%%%%%%%%%%%%%%%%%%%%%%%%%%%%%%%%%%%%%%
%%%%%%%%%%%%%%%%%%%%%%%%%%%%%%%%%%%%%%%%%%%%%%%%%%%%%%%%%%%%
\begin{document}
%% Capa
\begin{figure}[!h]
	\centering
	\includegraphics[scale=0.45]{logo/ufs_noframe.png}
\end{figure}

\begin{center}
	\textbf{ SERVIÇO PÚBLICO FEDERAL\\
		UNIVERSIDADE FEDERAL DE SERGIPE\\
		PRÓ-REITORIA DE PÓS-GRADUAÇÃO E PESQUISA}
	\vspace{10mm}

	PROGRAMA DE INICIAÇÃO CIENTÍFICA - PIBICVOL

	\vspace{15mm}

	%Título do plano de trabalho
	\textbf{\Large IMPLEMENTAÇÃO DE OPERADORES DE PROCESSAMENTO DE IMAGENS EM OPENCL }

	%% Descomentar em caso de IC Departamental
	%% ddantas 25/11/2024
	%\vspace{15mm}
	%
	%Discente: Ariel Lima Abade Bandeira
	%
	%Orientador: Daniel Oliveira Dantas
	%% FIM Descomentar em caso de IC Departamental

	\vspace{15mm}

	%% Usar áreas do conhecimento do CNPq conforme a tabela abaixo
	%% https://www.dropbox.com/scl/fi/1fxb483lnt57x2oo9cn11/areas_CNPq.pdf
	%% ddantas 06/08/2021
	Área do conhecimento: Ciência da Computação\\
	Subárea do conhecimento: Metodologia e Técnicas da Computação \\
	Especialidade do conhecimento: Processamento Gráfico (Graphics) \\

	\vspace{19mm}

	Relatório Final\\
	%% Caso haja bolsa, descomentar a linha abaixo.
	%% ddantas 19/08/2024
	%Período da bolsa: de (mês e ano) a (mês e ano)
    Período: de setembro de 2024 a agosto de 2025

	\vspace{10mm}

	{\large Este projeto é desenvolvido com/sem bolsa de iniciação científica}
	\vspace{10mm}

	\large{
		%% Descomentar a linha correspondente ao tipo de bolsa.


		%% ddantas 19/08/2024


		%PIBIC/COPES\\


		%\\ou\\


		%PIBIC/CNPq


		%\\ou\\


		%PIBIC/FAPITEC


		%\\ou\\
		PIBICVOL\\
		%\\ou\\


		%Iniciação Científica Departamental\\
	}
\end{center}

\newpage

%% Sumário
\begin{flushleft}
	\tableofcontents
\end{flushleft}

\newpage

%%%%%%%%%%%%%%%%%%%%%%%%%%%%%%%%%%%%%%%%%%%%%%%%%%%%%%%%%%%%
%%%%%%%%%%%%%%%%%%%%%%%%%%%%%%%%%%%%%%%%%%%%%%%%%%%%%%%%%%%%
%%%%%%%%%%%%%%%%%%%%%%%%%%%%%%%%%%%%%%%%%%%%%%%%%%%%%%%%%%%%
\section{Introdução}

O aumento da demanda por maior desempenho computacional, especialmente em áreas que requer processamento
de grande volume de dados em diversas estruturas, levou ao crescente uso de arquiteturas heterogêneas como
CPUs (do inglês \textit{Central Processing Unit}), GPUs (do inglês \textit{Graphics Processing Unit}) e
aceleradores diversos. A diversidade de arquiteturas contém desafios para portabilidade de ferramentas computacionais.

O \textit{framework} proprietário CUDA~\cite{cuda2008} promove uma API (do inglês \textit{Application Programming Interface})
para desenvolvimento de programas com processamento paralelo em dispositivos de GPU específicos da fabricante NVIDIA.
Utilizaremos a linguagem de programação Julia como o pacote CUDA.jl~\cite{besard2018juliagpu} para produzir algoritmos de processamento de imagens.

O \textit{framework} SYCL~\cite{sycl2020} apresenta uma camada de abstração de que permite o
desenvolvimento de programas para processamento heterogêneo na linguagem C++ com APIs para trabalhar CPUs, GPUs e FPGAs.
Em nosso estudo utilizamos AdaptiveCPP~\cite{acpp}, uma implementação da especificação SYCL, para analisar a
viabilidade para processamento de imagens médicas.

A ferramenta MATLAB desenvolvida pela \citeauthor{matlab2025} utiliza um linguagem programação
própria que permite a criação de algoritmos e modelos matemáticos e conta com funções integradas
para processamento de imagens em GPUs NVIDIA.

A biblioteca OpenCV~\cite{opencv2000} conta com funções integradas para processamento de imagens e manipulação de imagens
em CPU, mas também conta com estruturas para processamento em GPUs utilizando OpenCL e CUDA. Utilizaremos a biblioteca pors
meio da linguagem de programação Python na versão 3 para aplicarmos os algoritmos desenvolvidos.

A biblioteca VisionGL~\cite{visiongl} expõe funções para processamento paralelo de imagens 2D e 3D~\cite{Dantas2015opencl}
utiliza o \textit{framework} OpenCL~\cite{opencl2025} para acesso a arquiteturas heterogêneas compatíveis.

Utilizaremos os algorítmos apresentados em \citeauthor{Dantas2015opencl} para compararmos o desempenho dos operadores de
processamento de imagens e os resultados obtidos.

%%%%%%%%%%%%%%%%%%%%%%%%%%%%%%%%%%%%%%%%%%%%%%%%%%%%%%%%%%%%
%%%%%%%%%%%%%%%%%%%%%%%%%%%%%%%%%%%%%%%%%%%%%%%%%%%%%%%%%%%%
%%%%%%%%%%%%%%%%%%%%%%%%%%%%%%%%%%%%%%%%%%%%%%%%%%%%%%%%%%%%
\section{Objetivos}

\subsection{Objetivos Gerais}

O objetivo deste estudo é ampliar o suporte ao processamento de imagens da biblioteca VisionGL
ao verificar a viabilidade de outras ferramentas para processamento de imagens.

\subsection{Objetivos Específicos}

\begin{enumerate}
	\item Desenvolver operadores existentes na biblioteca VisionGL em SYCL com AdaptiveCPP e Cuda com Julia.
	\item Desenvolver benchmarks para VisionGL, SYCL, CUDA, OpenCV e MATLAB.
	\item Comparar a forma como foi implementada os operadores em cada ferramenta.
	\item Analisar o desempenho ao aplicar o benchmark na mesma imagem.
\end{enumerate}

%%%%%%%%%%%%%%%%%%%%%%%%%%%%%%%%%%%%%%%%%%%%%%%%%%%%%%%%%%%%
%%%%%%%%%%%%%%%%%%%%%%%%%%%%%%%%%%%%%%%%%%%%%%%%%%%%%%%%%%%%
%%%%%%%%%%%%%%%%%%%%%%%%%%%%%%%%%%%%%%%%%%%%%%%%%%%%%%%%%%%%
\section{Metodologia}

%%%%%%%%%%%%%%%%%%%%%%%%%%%%%%%%%%%%%%%%%%%%%%%%%%%%%%%%%%%%
%%%%%%%%%%%%%%%%%%%%%%%%%%%%%%%%%%%%%%%%%%%%%%%%%%%%%%%%%%%%
\subsection{Desenvolvimento dos operadores}

Nesta etapa, escolhemos algoritmos de processamento de imagens existentes na
VisionGL para adaptação dos conceitos originais para as ferramentas que buscamos
analizar.

\begin{itemize}
	\item operador de filtro desfoque gaussiano
	\item operador de convolução
	\item operadores ponto a ponto: inversão, escala de cinza e \textit{thresholding}
	\item operadores morfológicos: dilatação e erosão
\end{itemize}

Além disso, é necessário analisar a transferência de dados entre a memória do
dispositivo e a memória do hospedeiro (\textit{upload}/\textit{download}), bem
como o gerenciamento de memória para garantir que os dados sejam acessados de forma
eficiente durante a execução dos \textit{kernels}.

\subsection{Resolução}

O projeto foi interrompido antes de sua conclusão, portanto não foi possível
realizar a comparação de desempenho entre os operadores SYCL e os operadores
OpenCL da VisionGL. No entanto, a implementação dos operadores SYCL foi
concluída e está disponível no repositório do projeto.

% FEITO
%%%%%%%%%%%%%%%%%%%%%%%%%%%%%%%%%%%%%%%%%%%%%%%%%%%%%%%%%%%%
%%%%%%%%%%%%%%%%%%%%%%%%%%%%%%%%%%%%%%%%%%%%%%%%%%%%%%%%%%%%
%%%%%%%%%%%%%%%%%%%%%%%%%%%%%%%%%%%%%%%%%%%%%%%%%%%%%%%%%%%%
\section{Perspectivas de futuros trabalhos}

O projeto foi interrompido antes de sua conclusão, o que abre a possibilidade
de continuação em pesquisas futuras. Entre as próximas etapas previstas destacam-se:
\begin{enumerate}
	\item Comparação de operadores existentes na VisionGL com os operadores SYCL.
	\item Implementação de operadores SYCL compatíveis com a biblioteca VisionGL.
\end{enumerate}

% FEITO
%%%%%%%%%%%%%%%%%%%%%%%%%%%%%%%%%%%%%%%%%%%%%%%%%%%%%%%%%%%%
%%%%%%%%%%%%%%%%%%%%%%%%%%%%%%%%%%%%%%%%%%%%%%%%%%%%%%%%%%%%
%%%%%%%%%%%%%%%%%%%%%%%%%%%%%%%%%%%%%%%%%%%%%%%%%%%%%%%%%%%%
\section{JUSTIFICATIVA DE ALTERAÇÃO NO PLANO DE TRABALHO}

O plano de trabalho foi modificado devido o interesse em outras ferramentas de
processamento paralelo que poderiam trazer outros horizontes para a biblioteca
VisionGL.

\bibliography{refs.bib}
\nocite{visiongl}
\end{document}
